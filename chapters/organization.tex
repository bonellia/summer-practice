\chapter{Organization}
% Your observations and critiques about the company
In this chapter, I will be sharing my observations and critiques about the 
company.
\section{Organization and Structure}
Siren Bilişim was established about 10 years ago by METU CEng alumnis, namely 
Mustafa, Haldun and Koray. They have served various private and -mostly- 
government organizations, offering solutions involving embedded systems 
development initially. With the changing need and demand of the customers, 
nowadays they mostly develop systems for organizations like Securitas (an 
international security firm), EnerjiSA (a Turkish energy distribution company) 
and Turkish official organizations such as Ministry of Health, Agriculture and 
Forestry. They have 4 computer engineers as well as a data analyst and an 
outsourced designer. However, all three owners of the company actively take 
part in development activities. The company also provides consultancy services.
\par
My main observation regarding Siren Bilişim was to see that they didn't 
believe in numbers. With limited man-power, they had somehow managed to meet 
the needs of different big organizations. When I interacted with my peers 
during my time there, I have learned that they were still sparing time 
for pair programming, code reviews and workshops. During my time there, 
we had three workshops. I had chance to be familiar with n layer architecture 
and more during these workshops. On top that, given that they weren't on a 
tight schedule, I could arrange a meeting and get feedback for the source 
code I had written.