\chapter{Organization}
% Your observations and critiques about the company
In this chapter, I will be sharing my observations and critiques about the 
company.
\section{Organization and Structure}
Siren Bilişim was established about 10 years ago by METU CEng alumni, namely 
Mustafa, Haldun and Koray. They have served various private and -mostly- 
government organizations, offering solutions involving embedded systems 
development initially. With the changing need and demand of the customers, 
nowadays they mostly develop systems for organizations like Securitas (an 
international security firm), EnerjiSA (a Turkish energy distribution company) 
and Turkish official organizations such as Ministry of Health, Agriculture and 
Forestry. They have 4 computer engineers as well as a data analyst and an 
outsourced designer. However, all three owners of the company actively take 
part in development activities. The company also provides consultancy services.
\par
My main observation regarding Siren Bilişim was to see that they didn't 
believe in numbers. With limited man-power, they had somehow managed to meet 
the needs of different big organizations. When I interacted with my peers 
during my time there, I have learned that they were still sparing time 
for pair programming, code reviews and workshops. During my time there, 
we had three workshops. I had chance to be familiar with n layer architecture 
and more during these workshops. On top that, given that they weren't on a 
tight schedule, I could arrange a meeting and get feedback for the source 
code I had written.

\section{Methodologies and Strategies Used in the Company}
Before I have started my time there, the company was already having decent 
amount of meetings. However, most of them were kinda forced due to pandemic. 
After proposing a daily meeting which would be similar to scrums that are 
commonly seen on Agile methodology, Mustafa fortunately agreed. We had 
brief meetings daily to summarize our work and share our daily plan. These 
meetings usually took about 10-15 minutes.
\par
Although some of it depended on customer's preferences, the company mostly 
uses TFS, Azure DevOps to keep track of work done by its staff. For version 
control we have used Git which was quite useful. I kept track of my own 
daily work in a sheet and made sure that I had regular commits on the 
repository I was contributing in.
\par
I have shared my tendencies when it comes to requirements and actively 
communicated about the needs and constraints whenever I has demanded a work. 
In that regarding although documentation was not done thoroughly, I can still 
say that I was given enough instructions when I was expected certain work. 
When I felt stuck, I would contact either Derviş or Koray, but I was not 
limited to work with them alone. My peer Şahin would also join me on debugging 
when he was available. When it comes to picking a solution for certain tasks 
such as forms, Emin would share what he uses on their projects etc.