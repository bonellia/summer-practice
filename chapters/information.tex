\chapter{Information about Project}
% While writing about these phases, you should include also:  
% • problems encountered,  
% • their solutions found in the organization,  
% • methods followed,  
% • evaluation  of the project management,  
% • key points you have learned.
In this chapter, I will be explaining technical details regarding the 
project. To provide better context, I shall be introducing the project first. 
\par
PGMaster is an admininstration tool that mainly intends to help database and 
system administrators. PGMaster provides an interface for the systems that 
utilize containers which are homes to databases. Using Docker containers, our 
system creates and manages clusters. These clusters offer capabilities like 
high availability and data integrity. With various configurations, databases 
can be set up with modes like master, standby and more. Intended users of the 
product is admins that manage systems with high amounts of transactions. 
Whether it is private or official organization, many admins perform high level 
database management operations using tools on command-line interface (CLI). 
PGMaster can be considered as an abstraction over such interfaces as it allows 
critical operations to be initiated, observed and completed using a graphical 
user interface (GUI).
\par
PGMaster, although trivializing many tasks for admins, provides a CLI over the 
web application as well. This is achieved using WebSockets as well as various 
libraries which will be explained further in detail later on.
\par
Without any due, I will get into the phases I have been through during my 
internship.
\section{Analysis Phase}
On my first day of the internship, after being given a computer and setting up 
my development environment, I have been introduced to Derviş. Prior to my 
involvement, he was one and only contributor of the project. It was determined 
that I was going to implement frontend application of the system. After 
discussing the high level requirements of the system, it was clear to be that 
library that I was going to use for the project, namely Reactjs, was suitable. 
The project didn't have any document prior to implementation.
\par
Before my involvement, Derviş had already implemented some of the backend 
application modules which was using Java and Spring Boot framework. Being an 
experienced software developer (with 20+ years of experience), he was quite 
comfortable implementing Java and bash code. He also had extensive knowledge 
on PostgreSQL and database management systems in general. However, he needed 
a hand with the frontend application. There was only a JavaServer Pages (JSP) 
implementation with a form and some buttons written in HTML which were not 
sufficient for more sophisticated and dynamic nature of the system.
\par
We have discussed the project briefly and I proceeded to learn Reactjs the 
following days. I will be telling about implementation related details that 
were discussed on initial meetings on following sections. After understanding 
the general requirements of the system I have spent time analyzing the 
following:
\begin{itemize}
    \item Nature of the data to be processed.
    \item Design related constraints.
    \item Candidate modules and external interfaces.
    \item Documentation needs.
    \item Conventions to be followed during implementation.
\end{itemize}

I will try to explain considerations I have had for each of these items.


\subsection{Data}
\subsection{Design Constraints}
\subsection{Modules and Interfaces}
\subsection{Documentation}

Having had an introduction to software requirements specification (SRS) and 
software development lifecycle in general in our Software Engineering course, 
I usually try and understand the needs of any development activity prior to 
actual work. Aiming to follow the same trend here, I must admit that I was 
mildly disappointed, since although we didn't have a strict timeline to 
deliver the proof of concept (PoC) version of the application, the project 
was solely maintained and developed by one person. This should have 
-expectedly- obscured the need for any form of documentation.
\par I have actively communicated with my supervisor (Koray) and colleague 
(Derviş) on the subject and I was given permission (and responsibility) to 
write SRS for the project. I did not start writing the document right away, 
because it took me a while to fully grasp the overall needs and constraints 
of the system and the rest ofthe requirements naturally needs further 
discussion with stakeholders and the potential users. I still had chance to 
write some of the document by myself and it will be attached to the report. 
Final version of the SRS is expected to be completed after it is presented 
to The Ministry of Health.

\subsection{Documentation}

\section{Design Phase}
\section{Implementation Phase}
\section{Testing Phase}